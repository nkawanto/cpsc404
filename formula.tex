\documentclass{article}
\usepackage[utf8]{inputenc}
\usepackage{mathtools}
\DeclarePairedDelimiter\ceil{\lceil}{\rceil}
\DeclarePairedDelimiter\floor{\lfloor}{\rfloor}

\title{404}
\author{nkawanto }
\date{February 2021}

\begin{document}

\maketitle

\section{Lecture 1: Intro and Storage}
\begin{itemize}
\item $seek_{avg}$:
    \subitem $1 + \displaystyle\frac{avg\ tracks\ to\ sweep\ over}{\#cylinder}, avg = \frac{1}{3}tracks$, something to do with pascal triangle
\item {Rot. latency ($\gamma$):}
    \begin{itemize}
        \item $\gamma_{min} = 0$
        \item $\gamma_{max} = 1/RPM$
        \item $\gamma_{avg} = 1/2 max$
    \end{itemize}
\item Transfer (R/W a block) time:
    \subitem $\displaystyle\frac{\#sectors/page}{\#sectors/track}(\gamma_{max})$
\item Read time:
    \begin{itemize}
        \item random = seek + rotational delay + transfer for each page
        \item sequential = (seek + rotational delay)[first page] + (transfer)[all pages]
    \end{itemize}
\end{itemize}

\section{Lecture 2: External sorting}
\begin{itemize}
\item Merge Sort Overview:
    \begin{itemize}
    \item Sort phase: 
        \begin{itemize}
            \item read in B pages of data each time, sort internally (in memory)
            \item suppose entire data is in $m \cdot B$ pages
            \item sort phase produces m \textbf{sorted runs (sublists)} of B pages each
            \item each \textbf{sorted sublist (SSL)} is locally sorted
            \item (we are left with having to merge those m SSLs into single sorted file)
        \end{itemize}
    \item Merge phase: 
        \begin{itemize}
            \item repeatedly merge sorted runs
            \item pass 0: sort phase
            \item pass 1: produce m/2 sorted runs of 2B pages each
            \item pass 2: produce m/4 sorted runs of 4B pages each
            \item pass i: produce m/i sorted runs of $i \cdot B$ pages each
            \item ...
            \item continue until 1 sorted run of $m \cdot B$ pages produced
            \item 2-way merge can be optimized into k-way merge, where $size(k) <= size(B) - 1$
        \end{itemize}
    \end{itemize}
\item SSLS = $\displaystyle\floor*{\frac{\#file\ pages}{\#RAM\ pages}}$
\item "vanilla" mergesort still bad ($log_{2}n$ passes)
\item 2PMMS: 2 reads + 2 writes per block (still dont understand)
    \subitem $access_{avg}$ = seek + rot.delay + transfer
\item improving  runtime of TPMMS:
    \begin{itemize}
        \item group blocks by cylinder (cylindrification) (y)
        \item prefetching and double buffering:
            \begin{itemize}
                \item seek and rot. latency are made negligible for phase 1 (thanks to cylindrification) and are minimized for phase 2 (thanks for prefetching)
                \item total time for each pass for phase 2 $\simeq$ 1 R + 1 W
            \end{itemize}
    \end{itemize}
\item Estimate IO cost $==$ \# page IOs incurred:
    \begin{itemize}
        \item block (page) size = 4K
        \item filesize: 1GB $\rightarrow$ $\frac{1000\cdot 2^{10}}{4\cdot 2^{10}} = 256000 pages$
        \item RAM = 50MB
        \item phase 1 (sort): $256000\ pages \cdot (R+W)$
        \item phase 2 (merge): $256000\ pages \cdot (R+W)$
        \item total= 4 $\cdot$ 256000 IOs
    \end{itemize}
\item Estimate elapsed IO time
    \begin{itemize}
        \item $S_{avg} = 8ms$
        \item $\gamma_{avg} = 7ms$
        \item $transfer = 0.5ms$
        \item Elapsed time = $4\cdot 256000\cdot (8+7 +0.5)ms$
    \end{itemize}
\item Estimate elapsed IO time (minimizing IO)
    \begin{itemize}
        \item Sort phase:
            \begin{itemize}
                $(R+W)\cdot \#$ pages in file + $(rand.\ accesses\ for\ every\ B\ pages\ (\gamma_{avg} + S_{avg}))[negligible]$
            \end{itemize}
        \item Merge phase:
            \begin{itemize}
                \item B = $\frac{50MB RAM}{4K\ pages} = 12800\ pages$
                \item SSL = 20
                \item random reads = $\floor*{\frac{12800}{20}} = 609$
                \item random writes = $12800-20\cdot 609$
                \item $\displaystyle{256000\cdot (R+W)\cdot 0.5ms+{\ceil*{\frac{256000}{609}} + \ceil*{\frac{256000}{620}}}}$
            \end{itemize}
    \end{itemize}
\end{itemize}

\section{Lecture 3: B+Trees Basics}
\begin{itemize}
    \item Data entries:
        \begin{itemize}
            \item rid: record id
            \item k as key value, denote its corresponding entry as k*
            \item alt1: k* = data with record value k
            \item alt2: k* = <k, rid of data record with search key key k>
                \subitem k points to rid and entry
            \item alt3: k* = <k, list<rids of data records with search key k>[]>
                \subitem k points to list<rid, entry>
        \end{itemize}
    \item 1 node of B+Tree = 1 page
\end{itemize}

\section{Lecture 4: B+Trees}
\begin{itemize}
    \item B+Tree structure
        \begin{itemize}
            \item internal node: <key, pointer> pair
            \item An internal node has m <key, pointer> pairs + 1 additional pointer
            \item Insert/delete cost $log_{F}N$
                \subitem F = avg fanout (i.e. \# children), tends to be in hundreds; N = \# leaf pages, storing data entries
            \item Minimum 50\% occupancy (except for root), each internal node contains $d\leq m\leq 2d$ entries. $d$ is order of tree
            \item Each leaf has to be at least half full
            \item Index entry (held by internal node): <key, pointer>
            \item Leaf page (leaf node): contains $d$ data entries (can be alt1/alt2/alt3)
            \item  
            \item 1 node of B+Tree = 1 page
            \item Typical order: 100 ($d$ = 100). means for each internal node (minus root), $100\leq \#entries\leq 2\cdot 100$.
            \item Fill factor = \# entries in a node/max possible
                \subitem \# entries in a node/2d, where d = order of B+Tree
            \item Average fill-factor = ln(2) = 66.5\% = 0.665
                \subitem $\rightarrow$ average fanout $\simeq$ $2d \cdot ln(2) = 133$
            \item root is $1 \leq \#entries \leq 2d$.
        \end{itemize}
    \item max \# <key,pointer> pairs/page?
        \subitem $\floor*{\frac{pagesize}{(keysize + ptrsize)}}$
    \item Remember that average fanout = 133
    \item Typical capacities: 
        \begin{itemize}
            \item Level 1: 1 page = 8KB (assuming 8K/page)
            \item Level 2: 133 pages = 1MB (approx.)
            \item Level 3: $133^2$ pages = 133MB (approx.)
            \item Level n: $\displaystyle{133^{n-1}}$ pages
        \end{itemize}
    \item Min B+Tree height:
        \begin{itemize}
            \item 4K block, 12b key, 8b ptr. For a file occupying b (logical) blocks, find min/max/avg height of B+tree index
            \item $\floor*{\frac{4096}{12+8}} = 204$
            \item min height = $\ceil*{log_{204+1}b}$
            \item max height = $\ceil*{log_{204+1}b} + 1$
            \item avg height = $\displaystyle{\ceil*{log_{(204 + 1)\cdot 0.665}(b/0.665)}}$
        \end{itemize}
\end{itemize}

\section{Lecture 4: B+Trees Insert/Delete}
\begin{itemize}
    \item Not much important here, just insert/delete mechanism
    \item https://www.cs.usfca.edu/~galles/visualization/BPlusTree.html
    \item Order = 1/2(max \# of nodes)
    \item Root must have at least one entry and 2 pointers
    \item Repeating last lecture, every internal node (minus root): $d \leq entries \leq 2d$ and leaf nodes must be $\geq$ 50\% full
    \item Insertion enforces this occupancy constraint as DE gets added to B+Tree
    \item Merge (from deletion) causes an entry to be deleted from parent, merging propagates up recursively
    \item In extreme case, tree may become 1 level shorter.
\end{itemize}

\section{Lecture 5: B+Trees Optimizations}
\begin{itemize}
    \item Prefix Key Compression
    \begin{itemize}
        \item Important to increase fanout
            \subitem search key "david" and "devarakonda", "david" can be abbreviated into "dav", shorten keys but keeping them unique
    \end{itemize}
    \item Bulk loading
        \begin{itemize}
            \item Option 1: repeated insertions
            \begin{itemize}
                \item Slow
                \item Does not ensure sequential storage of leaves
            \end{itemize}
            \item Option 2: bulk loading
            \begin{itemize}
                \item Has advantages for concurrency control
                \item Fewer IOs during build
                \item Leaves will be stored sequentially (and linked)
                \item Can control "fill factor" on pages
            \end{itemize}
        \end{itemize}
        \item Insert/deletes leave tree height-balanced: $O(log_F N)$ cost
        \item High fanout means depth rarely more than 4 or 5
        \item Almost always better than simply maintaining a sorted filesize
        \item Key compression increases fanout, reduces height
        \item Bulk loading can be much faster than repeated inserts for large dataset
        \item Widely used index in DBMS because of its versatility; one of the most optimized components of a DBMS
\end{itemize}

\section{Lecture 6: Static hashing}
\begin{itemize}
    \item Hash-based indexes are best for equality selections
        \subitem no traversal; direct computation where DE k* should be, given value k.
    \item Cannot support range searches
    \item Static vs dynamic hashing:
    \begin{itemize}
        \item static hashing assumes file doesn't change/doesn't change much
        \item dynamic hashing specifically designed to handle files that undergo many and frequent updates
    \end{itemize}
    \item Static hashing:
    \begin{itemize}
        \item primary pages fixed, allocated sequentially, never de-allocated; overflow pages (imagine Java hash<linkedlist>) used if needed.
        \item $h(k)\ mod\ M$ = bucket to which DE with key k (i.e. k*) belongs. M = \# of buckets
        \item [[page, page, page],[page],[page],...]
        \item Buckets contain DE (alt1, alt2, alt3)
        \item Long oferflow chains can develop and degrade performance (when update distribution is skewed). What would happen if we insert many records with keys hashed to $MOD\ M$?
            \subitem Extendible Hashing (EH) and Linear Hashing (LH): two major dynamic hashing techniques to fix this problem.
        \item both EH and LH attempt to "techniques the buckets to avoid or minimize overflow.
    \end{itemize}
\end{itemize}

\section{Lecture 7: Dynamic hashing}
\begin{itemize}
    \item Extendible Hashing:
    \begin{itemize}
        \item Motivation: avoid/minimize overflow upon insertions. Reorganize (i.e. rehash) file by, say, doubling \# of buckets?
            \subitem R/W all pages is expensive!
        \item Idea: use directory of pointers to buckets (think dirent)
            \subitem static hashing has no directory
        \item We double the logical \# of buckets by doubling directory; splitting just the bucket that overflowed.
    \end{itemize}
    \item If directory fits in memory, equality search answered with one disk access; else two (for alt1 DE).
    \item Directory can grow large if distribution of hash values is skewed.
    \item Delete: if removal of DE makes bucket empty,
    \begin{itemize}
        \item check to see if all 'split images' can be merged (if a bucket is merged with its split image, we should decrement its Local Directory by 1)
        \item if each directory elements point to the same bucket as its split image, can halve directory
        \item rarely done in practice (leave room for future insertions)
    \end{itemize}
    \item \textbf{need to read more on this}
\end{itemize}

\end{document}
